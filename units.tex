\documentclass{revtex4}
\usepackage{amsmath,amssymb}

\begin{document}
We introduce dimensionless time, space and field variables
\begin{equation}
  \bar{x} = \mu x \qquad \bar{t} = \mu t \qquad \bar{\phi} = \frac{\phi}{\Lambda} \, .
\end{equation}
The dimensionless potential can be derived either by rewriting the action or directly by rewriting the equations of motion in terms of $\bar{x}$,$\bar{t}$ and $\bar{\phi}$
\begin{equation}
  \tilde{V}(\phi) = \frac{V(\phi)}{\Lambda^2 \mu^2}
\end{equation}
which allows us to determine the relationship between the dimensionless potential parameters in the code and the parameters appearing in the potential.
Notice that $\tilde{V}$ is a function of $\phi$, not $\bar{\phi}$ in the above.
To obtain the functional form of the dimensionless potential in terms of $\bar{\phi}$, which we denote $\bar{V}(\bar{\phi})$, we write $\phi = \Lambda\bar{\phi}$ and absorbing the $\Lambda$'s into the definitions of dimensionless potential parameters in the code.
The dimensionless scalar equation of motion used in the code is then
\begin{equation}
  \frac{\partial^2\bar{\phi}}{\partial\bar{t}^2} + 3\frac{H}{\mu}\frac{\partial\bar{\phi}}{\partial\bar{t}} + \frac{\partial\bar{V}(\bar{\phi})}{\partial\bar{\phi}} = 0
\end{equation}

For example, consider a monomial potential
\begin{equation}
  V(\phi) = g_\alpha \phi^\alpha = g_\alpha\Lambda^\alpha\left(\frac{\phi}{\Lambda}\right)^\alpha
\end{equation}
so that in the code, the dimensionless parameter $\bar{g}_\alpha$ is
\begin{equation}
  \bar{g}_{\alpha} = g_{\alpha}\frac{\Lambda^{\alpha-2}}{\mu^2} \, .
\end{equation}
For convenience in determining the evolution of the Hubble constant $H$, the lattice code defines $\Lambda = M_{\rm P}$ with $M_{\rm P}$ the reduced Planck mass.  We therefore obtain
\begin{equation}
  \bar{g}_{\alpha} = g_{\alpha}M_{\rm P}^{\alpha-2}\mu^{-2}
\end{equation}
and
\begin{equation}
  \bar{V}(\bar{\phi}) = \bar{g}_{\alpha}\bar{\phi}^\alpha
\end{equation}
An analogous procedure works for arbitrary functional forms of the potential.
Similarly, if we know the physical value of a potential parameters (say in $\lambda\phi^4$ or $m^2\phi^2$), we can use this to obtain the value of $\mu$ implicitly used in the code.

\end{document}
